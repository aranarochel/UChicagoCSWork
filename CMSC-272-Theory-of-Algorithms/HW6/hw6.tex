%Do not change 
\documentclass[12pt, oneside]{article}
\usepackage{amssymb,amsmath}
\usepackage[margin=1in]{geometry}
\usepackage{textpos}
\usepackage{amsthm}
\usepackage{amsfonts}
\usepackage{graphicx}


% You may add the packages you need here



\begin{document}
%Do not modify
\begin{textblock*}{4cm}(-1.7cm,-2.3cm)
\noindent {\scriptsize CMSC 27200 Spring 2016} 
\end{textblock*}

%Do not modify other than putting your name where stated
\begin{textblock*}{8cm}(12.5cm,-2.3cm)
\noindent {Name: Jaime Arana-Rochel} 
\end{textblock*}


\vspace{1cm}

\makeatletter
\setlength{\@fptop}{0pt}
\makeatother

%Do not modify other than typing the homework number after #
\begin{center}
\textbf{\Large Homework \#6}
\end{center}


%Rest should contain your solution for the homework. Feel free to improvise in ways that you believe make grading easier.
\subsection*{Q1: Exercise 6 Chapter 7}
\subsubsection*{Algorithm}
\begin{verbatim}
// function to create a bipartite graph from a floor plan
create_bipartite:
   // we create the two vertex sets
   // a node x_i corresponds to switch i
   Add n switch nodes to set X
   
   // a node y_j corresponds to fixture j
   Add n fixture nodes to set Y
   
   // we add edges to set E
   For every x_i in X:
      For every y_j in Y:
         For every wall in the floor plan:
            test whether (x_i, y_j) intersects wall
         
         If (x_i, y_j) did no intersect any wall:
            add (x_i, y_j) to edge set E
   
   return graph B(X,Y,E)
end


Ergonomic_Test:
   // we create our bipartite graph
   B = create_bipartite()
   
   // create graph B' to pass onto Ford-Fulkerson
   B' = 
     Connect node s to every node in X from B
     Connect node t to every node in Y from B
     Add capacity 1 to every edge
   
   max_flow = Ford-Fulkerson(B')
   
   If max_flow = n:
      return "Ergonomic"
   else:
      return "not Ergonomic" 
end
\end{verbatim}

\subsubsection*{Correctness}
The algorithm has two parts. First it builds a bipartite graph $B = (X,Y,E)$. A node $x_i \in X$ corresponds to a switch $i$, and a node $y_j \in Y$ corresponds to a fixture $j$. We only add a pairing $(x_i,y_i)$ to the edge set $E$ only if it doesn't intersect any of the walls of the floor plan.\\\\
Next, we augment with graph $B$ to graph $B'$ so that we can pass it on to Ford-Fulkerson. This is done in the same way as in the book: adding nodes $s,t$ and having the capacity for each edge be 1. We get a maximum flow from Ford-Fulkerson and then check if it equals $n$. If it does, then we know that not only did we find a maximum matching in $B$, but we also found a perfect matching in $B$. This perfect matching correctly tells us that our floor plan is ergonomic since a perfect matching in $B$ is a pairing of the $n$ switches and fixtures. Since we only added an edge $e$ to $E$ if the switch-fixture pairing wasn't obstructed, this provides us with an ergonomic answer.\\

\subsubsection*{Running Time}
There are three $for$ loops in the function which creates our bipartite graph. It creates every possible switch-fixture pairing (from $n$ switches and fixtures) and then tests the line segment of this pairing against $m$ walls. Thus the running time of this function is $O(n^2m)$.\\
$Ergonomic$-$Test$ then runs Ford-Fulkerson to get a maximum flow, and then simply tests if the flow equals $n$. The graph $B$ could at worst have $n^2$ edges in E, so Ford-Fulkerson runs in $O(n*n^2) = O(n^3)$.\\
The total running time is then $O(n^3 + n^2m)$.\\

\subsection*{Q1: Exercise 7 Chapter 7}
\subsubsection*{Algorithm}
\begin{verbatim}
// function to create a flow network graph
create_network:
   
   // connect client and base station nodes
   For i = 1 to n:
      node u_i = client i
      Add u_i to node set V
      For j = 1 to k:
         node v_j = base station j
         Add v_j to V
         
         If the distance from u_i to v_j <= range parameter r:
            add edge (u_i,v_j) to set E with capacity 1
         
   Add sink node t to V
   Add source node s to V
   
   Add an edge from s to every client node with capacity 1
   Add an edge from every base station node to t with capacity L
   
   return G(V,E)
end

Client-Station-Check:
   Graph G = create_network()
   max_flow = Ford-Fulkerson(G)
   
   If max_flow = n:
      return "All clients connected"
   Else:
      return "Not all clients connected"
end
\end{verbatim}

\subsubsection*{Correctness}
The algorithm has two parts. First we build our flow network graph $G$. For every client $i$ we create a node $u_i$, and for every base station $j$, we create a node $v_j$. We create an edge $(u_i,v_j)$ with capacity 1 if the client $i$ and the base station $j$ are in range of each other. Lastly we connect our source node $s$ to each of the $u_i$ with capacity 1, and we connect each of the $v_j$ to the sink node $t$ with capacity $L$.\\\\
With this graph, we run Ford-Fulkerson and receive a maximum flow. We then make the following claim: We can connect all of the clients to a base station if our maximum $s$-$t$ flow in $G$ has value equal to $n$.\\\\
Proof: If we can connect every client to a base station, then Ford-Fulkerson will have pushed flow through every $s$-$t$ path containing each of the $(u_i,v_j)$ edges. Since each edge has capacity 1, then our flow has value $n$. We do not violate the capacity constraints on the graph at any point. Since the edges connecting the base stations to the sink $t$ have capacity $L$, Ford-Fulkerson ensures that we don't send flow greater than $L$ through each of these base stations. In other words, we never try to connect clients to a base station such that we violate the base station connection capacity $L$. Thus, the algorithm correctly tells whether each client can be connected.\\

\subsubsection*{Running Time}
Creating the flow network graph runs in time $O(nk)$ time due to the two $for$ loops. Then, our final running time is simply the time to execute Ford-Fulkerson on graph $G$. With $O(n+k)$ nodes and $O(nk)$ edges, that is a running time of $O((n+k)*nk) = O(n^2k + nk^2)$.


















\end{document}