%Do not change 
\documentclass[12pt, oneside]{article}
\usepackage{amssymb,amsmath}
\usepackage[margin=1in]{geometry}
\usepackage{textpos}
\usepackage{amsthm}
\usepackage{amsfonts}
\usepackage{graphicx}

% You may add the packages you need here



\begin{document}
%Do not modify
\begin{textblock*}{4cm}(-1.7cm,-2.3cm)
\noindent {\scriptsize GEOS 254 Winter 2016} 
\end{textblock*}

%Do not modify other than putting your name where stated
\begin{textblock*}{8cm}(12.5cm,-2.3cm)
\noindent {Name: Jaime Arana-Rochel} 
\end{textblock*}


\vspace{1cm}

\makeatletter
\setlength{\@fptop}{0pt}
\makeatother

%Do not modify other than typing the homework number after #
\begin{center}
\textbf{\Large Homework \#5}
\end{center}


%Rest should contain your solution for the homework. Feel free to improvise in ways that you believe make grading easier.
\subsection*{1) Integrating a Function}
Using the recursive trapezoidal rule to within a tolerance of $\epsilon=10^{-4} $, the integral of the function is approximated as $147.413177$. To obtain this integral threshold, I had to use a value of $n=13$, which consequently meant my stepsize was $h=\frac{5}{2^{13}}\approx0.000610$.


\subsection*{2) Simpson's Method}
Using Simpson's method, and using the same stepsize value for $h$ as above, the calculated integral I obtained was $147.413159$. This estimate is actually reasonably close to the real answer
\[e^5-1 \approx 147.413159102577\ldotp \]


\subsection*{3) More Integration}
To evaluate this integral so that the fifth decimal point is unchanging, I again used the same stepsize as above with the Simpson's method. The calculated integral using this method is $3.141592$.














\end{document}